%%%%%%%%%%%%%%%%%%%%%%%%%%%%%%%%%%%%%%%%%
% Beamer Presentation
% LaTeX Template
% Version 1.0 (10/11/12)
%
% This template has been downloaded from:
% http://www.LaTeXTemplates.com
%
% License:
% CC BY-NC-SA 3.0 (http://creativecommons.org/licenses/by-nc-sa/3.0/)
%
%%%%%%%%%%%%%%%%%%%%%%%%%%%%%%%%%%%%%%%%%

%----------------------------------------------------------------------------------------
%	PACKAGES AND THEMES
%----------------------------------------------------------------------------------------

\documentclass{beamer}

\mode<presentation> {

% The Beamer class comes with a number of default slide themes
% which change the colors and layouts of slides. Below this is a list
% of all the themes, uncomment each in turn to see what they look like.

%\usetheme{default}
%\usetheme{AnnArbor}
%\usetheme{Antibes}
%\usetheme{Bergen}
%\usetheme{Berkeley}
%\usetheme{Berlin}
%\usetheme{Boadilla}
%\usetheme{CambridgeUS}
%\usetheme{Copenhagen}
%\usetheme{Darmstadt}
%\usetheme{Dresden}
%\usetheme{Frankfurt}
%\usetheme{Goettingen}
%\usetheme{Hannover}
%\usetheme{Ilmenau}
%\usetheme{JuanLesPins}
%\usetheme{Luebeck}
\usetheme{Madrid}
%\usetheme{Malmoe}
%\usetheme{Marburg}
%\usetheme{Montpellier}
%\usetheme{PaloAlto}
%\usetheme{Pittsburgh}
%\usetheme{Rochester}
%\usetheme{Singapore}
%\usetheme{Szeged}
%\usetheme{Warsaw}

% As well as themes, the Beamer class has a number of color themes
% for any slide theme. Uncomment each of these in turn to see how it
% changes the colors of your current slide theme.

%\usecolortheme{albatross}
%\usecolortheme{beaver}
%\usecolortheme{beetle}
%\usecolortheme{crane}
%\usecolortheme{dolphin}
%\usecolortheme{dove}
%\usecolortheme{fly}
%\usecolortheme{lily}
%\usecolortheme{orchid}
%\usecolortheme{rose}
%\usecolortheme{seagull}
%\usecolortheme{seahorse}
%\usecolortheme{whale}
%\usecolortheme{wolverine}

%\setbeamertemplate{footline} % To remove the footer line in all slides uncomment this line
%\setbeamertemplate{footline}[page number] % To replace the footer line in all slides with a simple slide count uncomment this line

%\setbeamertemplate{navigation symbols}{} % To remove the navigation symbols from the bottom of all slides uncomment this line
}

\usepackage{mathtools,amsmath,amsfonts}
\usepackage{graphicx} % Allows including images
\usepackage{booktabs} % Allows the use of \toprule, \midrule and \bottomrule in tables

%----------------------------------------------------------------------------------------
%	TITLE PAGE
%----------------------------------------------------------------------------------------

\title[3D Sudokus]{3D Sudokus for SAT Solvers} % The short title appears at the bottom of every slide, the full title is only on the title page

\author{F. Boutkan, J. Barelds} % Your name
\institute[UvA] % Your institution as it will appear on the bottom of every slide, may be shorthand to save space
{
University of Amsterdam \\ % Your institution for the title page
}
\date{\today} % Date, can be changed to a custom date

\begin{document}

\begin{frame}
\titlepage % Print the title page as the first slide
\end{frame}

\begin{frame}
\frametitle{Overview} % Table of contents slide, comment this block out to remove it
\tableofcontents % Throughout your presentation, if you choose to use \section{} and \subsection{} commands, these will automatically be printed on this slide as an overview of your presentation
\end{frame}

%----------------------------------------------------------------------------------------
%	PRESENTATION SLIDES
%----------------------------------------------------------------------------------------

%------------------------------------------------
\section{What is a 3D Sudoku?} % Sections can be created in order to organize your presentation into discrete blocks, all sections and subsections are automatically printed in the table of contents as an overview of the talk
%------------------------------------------------

\begin{frame}
\frametitle{What is a 3D Sudoku?}

% TODO: image
\begin{itemize}
	\item A 3D Sudoku generalizes the 2D Sudoku by introducing a third dimension for sequences of cells called layers.
	\item A 3D Sudoku can be of any size $N$, but each row, column, and layer must be equal in length.
	\item Similar to the 2D Sudoku, every number $d \in \{1, \hdots, N\}$ must appear exactly once in each row, column, \emph{and} layer.
	\item Since blocks cannot exist in three dimensions, the block constraint is dropped.
\end{itemize}

\end{frame}

%------------------------------------------------

\section{Hypothesis}
\begin{frame}
\frametitle{Hypothesis}
\begin{itemize}
	\item The hypothesis of this paper can be formulated as follows. Given is a 3D Sudoku of size $N$ with $M$ filled cells. As $N$ grows, the complexity (i.e., the CPU time, restarts, decisions and conflicts) grows and the relative number of givens in proportion to $N$ must increase in order for the 3D Sudoku to be tractable.
\end{itemize}
\end{frame}

\section{Experimental Setup}
\begin{frame}
\frametitle{Experimental Setup}
\begin{itemize}
 \item Encoding adapted from Weber.
 \item Dataset generated with custom written 3D Sudoku generator using Freek's Algorithm.
 \item Since Freek's Algorithm generates a complete 3D sudoku, erase $N^3 - M$ cells in order to obtain the required $M$ filled cells.
 \item Convert the 3D Sudoku constraints to DIMACS CNF and give it to Minisat, which attempts to find a model that satisfies the constraints.
 \item Generate, convert, and solve 3D Sudokus for a range of parameters $(N \in \{3, \hdots, 15\}, M:N^3 = \{.1, .2, .3, \hdots, .8, .9\}, k = 5)$, where $k$ is the number of repetitions. Take the means of the results for each $(N, M)$ tuple.
\end{itemize}
\end{frame}

\section{Results}
\begin{frame}{Results}
    \begin{table}[]
\centering
\begin{tabular}{@{}l|lllllllllllll@{}}
\toprule
N & 3  & 4  & 5  & 6  & 7  & 8  & 9  & 10 & 11 & 12 & 13 & 14 & 15 \\ \midrule
$M_{low}$                      & \textbf{10} & \textbf{30} & \textbf{10} & \textbf{20} & \textbf{20} & 20 & \textbf{30} & \textbf{30} & 30 & 30 & 30 & 30 & 30 \\
$M_{up}$                       & 20 & 40 & 20 & 30 & 30 & \textbf{30} & 40 & 40 & 40 & 40 & \textbf{40} & \textbf{40} & \textbf{40}  
\end{tabular}
\caption{Different size of puzzles and the estimated interval (lower- and upperbound) for the percentage of givens needed to find a solution. A bold digit indicates that the estimate is closer to that percentage.}
\label{tab_res_sum}
\end{table}
\end{frame}

\section{Conclusion \& Further Work}
\begin{frame}{Conclusion \& Further Work}
\begin{itemize}
    \item As it turns out, the relative number of givens $M : N^3$ required for the 3D Sudoku to be tractable (and deterministic) increases as $N$ grows.
    \item Further work
        \begin{itemize}
            \item Could consist of pinpointing the exact value of $M$ for each $N$.
            \item Different methods of generating 3D Sudokus?
            \item How does this generalize to $n$-D Sudokus?
        \end{itemize}
\end{itemize}
\end{frame}

%------------------------------------------------

%------------------------------------------------

\begin{frame}
\Huge{\centerline{The End}}
\end{frame}

%----------------------------------------------------------------------------------------

\end{document}