\begin{abstract}
3D sudokus are a generalization of the classic 2D sudoku to three dimensions. We examine the relation between the relative number of prefilled cells needed to solve a 3D sudoku. The hypothesis is "as the size of a 3D sudoku grows, the relative number of givens needed to find a solution also increases". 3D sudoku's of diffent sizes and with varying degrees of prefilled cells were generated and encoded in CNF. Minisat was used to solve the puzzle. Conflict and decision statistics were used to determine if the puzzle could be solved solely by logical inferences. The hypothesis is accepted although future work could provide more insight in the exact number of givens needed.
\end{abstract}