\documentclass[11pt]{article}
%Gummi|065|=)
\usepackage{mathtools,amsmath,amsfonts}
\title{\textbf{Sudoku3D Propositional Logic Representation Notes}}
\author{Jochem Barelds}
\date{}
\begin{document}

\maketitle

A 3D sudoku is an $N \times N \times N$ structure (with $N \in \mathbb{N}$), where every row, colum and layer needs to contain exactly one digit $d \in \{1, \hdots, N\}$. A single row, column or layer may not use the same digit more than once. This can be expressed as (TODO: reference to Weber):

\begin{align}
	\text{valid}(x_1, \hdots, x_N) := \bigwedge_{d=1}^{N} \bigvee_{i=1}^{N} \bigwedge_{1 \leq d \leq d' \leq N} \neg x_i^{d} \vee \neg x_i^{d'} 
\end{align}

Then the constraints of the sudoku can be represented as:

TODO: resolve inconsistent representations of $x_i$ and $x_{i,j,k}$.

\begin{align}
\begin{split}
	\text{sudoku}(\{x_{i,j,k}\}) := \bigwedge_{k=1}^{N} (\bigwedge_{i=1}^{N} \text{valid}(x_{i,1,k}, \hdots, x_{i,N,k}) \wedge \bigwedge_{j=1}^{N} \text{valid}(x_{1,j,k}, \hdots, x_{N,j,k})) \\
	\wedge \bigwedge_{j=1}^{N} (\bigwedge_{k=1}^{N} \text{valid}(x_{1,j,k}, \hdots, x_{N,j,k}) \wedge \bigwedge_{i=1}^{N} \text{valid}(x_{i,j,1}, \hdots, x_{i,j,N}))
\end{split}
\end{align}

\end{document}
