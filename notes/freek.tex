\documentclass{article}
\usepackage{amsmath}
\usepackage[utf8]{inputenc}

\title{notes}
\author{Freek Boutkan}
\date{October 2017}

\begin{document}

\section{Encoding}
Since symbols in propositional logic can only be true or false it is not possible to encode the number of a cell with a single symbol because for a $N$ sized sudoku there are $N$ different posible digits for each cell. Thus for a single cell $N$ different symbols are needed. In this paper the sentence 'Position $x$, $y$, $z$ contains a $d$' is enconded as follows:
\begin{align}
p_{xyz}^d &= 1
\end{align}
We have adopted Weber encoding for this experiment. Webers encoding consists of the following elements
\begin{enumerate}
\item At least one of $N$ digits in a single cell. This can be encoded as a single clause of length $N$
\begin{align}
\bigvee_{d=1}^N p_{xyz}^d
\end{align}
\item At most one of $N$ digits in a single cell. This can be encoded as $\sum_{n=1}^{N-1}n$ binary clauses:
\begin{align}
\bigwedge_{1 \leq d < d' \leq N}^N \neg p_{xyz}^d \vee \neg p_{xyz}^{d'}
\end{align}
\item Validity of $N$ positions. Weber defines a row (or column or region) of $I$ cells as valid if all the cells contain distinct values. This is true because there are just as many digits as cells. This can be encoded as $N \cdot \sum_{n=1}^{N-1}n$ binary clauses:
\begin{align}
\text{Valid}(p_1,\dots,p_N) \implies \bigwedge_{1 \leq i < i' \leq N}\bigwedge_{d=1}^N \neg p_{i}^d \vee \neg p_{i'}^{d}
\end{align}
\end{enumerate}

\subsection{Number of clauses}
For each of the $N^3$ cells an 'At least one` and an 'At most one` clause is needed. Also for each dimension (row/column/layer) $N^2$ valid clauses.
Given an empty 3D sudoku of size $N$ the number of clauses $c$ needed to represent the puzzle is given by formula
\begin{align}
c(N) = N^3 + N^3 \cdot \sum_{n=1}^{N-1}n + 3N^3\sum_{n=1}^{N-1}n = N^3(1 + 4\sum_{n=1}^{N-1}n)
\end{align}
\begin{align}
l(N) = \frac{N\cdot N^3 + 2\cdot4\cdot N^3 \sum_{n=1}^{N-1}}{N^3(1 + 4\sum_{n=1}^{N-1})} = \frac{8N\sum_{n=1}^{N-1}}{4\sum_{n=1}^{N-1}+1}
\end{align}
\end{document}

